目前市面上之手機遊戲多以免費遊玩商業模式(Free-to-Play, F2P)為主,使得遊戲內購買( In-App Purchase, IAP)顯得越來越重要,已然成為遊戲開發商營運之重點,為了能夠推出成功吸引各式玩家的精準行銷,需要資料分析團隊針對付費玩家進行研究,並且希望能夠在新進玩家族群中,成功預測出潛在付費玩家,以利提升IAP的意願,因此,如何在付費玩家資料中,有效探勘出資料特徵並透過機器學習進行預測,則為此次研究的目標。

本論文對此議題提出一巨量資料探勘框架,將需先將資料進行前處理以及預測前之資料分析,隨後訓練機器學習與其最佳化處理,最後再依預測之結果導入資料特徵重要性分析之中,完成整體預測與分析之工作,此框架將由四大階段組成: (1) 資料前處理階段、(2) 資料分析階段、(3) 機器學習階段及(4) 預測結果分析階段。

根據實驗結果,藉由我們提出的巨量資料探勘框架,利用無價值玩家觀察期清理了無價值的資料,並藉由付費玩家定義期準備了付費玩家與非付費玩家目標值,利用資料特徵探勘期探勘出了有價值的玩家遊戲行為軌跡。透過探索性資料分析(Exploratory Data Analysis, EDA)找出不合理資料特徵與高資訊量資料特徵,推測出有價值的資料特徵。能夠經由學習模型之預測,預測出潛在之新進付費玩家,並依其預測結果,分析資料特徵重要性,了解到玩家消費原因與遊戲之連動性。整體來說,該框架將能使得預測付費玩家之時間成本與人力成本有效降低,並得到對於行銷有利的資訊。

關鍵字:付費預測、免費遊玩遊戲、巨量資料、資料探勘、機器學習、極限梯度提升