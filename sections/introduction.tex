\chapter{緒論}

\section{研究背景與動機}

對於許多遊戲商而言,準確預測玩家流失對於長期成功至關重要。近年來,手機遊戲商大多以免費遊戲為主,不再是以往的買斷制或月費制,在免費遊戲的模式之下,遊戲商之營收有非常顯著的成長,例如:「絕地要塞 2」 ( Team Fortress 2 ) 原本為買斷制型式遊戲,新收入的來源僅限於沒購買過遊戲的人,這樣的商業模式與他們的發展策略並不完全吻合,於是,在 2011 年改為免費遊戲,除了擴大玩家受眾,還透過遊戲不斷地更新保持玩家的興趣,最終,遊戲營收提高達 12 倍之多~\cite{miller2012gdc},但是,免費遊戲在定義是否為流失玩家上極為困難,一旦玩家想離開了隨時都能停止遊玩,沒有必要告訴遊戲商其決定,遊戲商也容易因此失去挽回玩家的機會。此外,在遊戲中,獲得一個新玩家的成本也比留住一個玩家要昂貴許多。

因此,精準地預測玩家流失,即使是微小地提升,也可能導致營收有顯著的提升。透過上述情境發想,若能透過機器學習訓練出一流失預測模型並運用於遊戲領域,利用其預測結果,並從中了解玩家流失的原因與動機,將可以交由運營人員作為後續挽留玩家的操作依據,以強化留存並進一步增加營收。

\section{研究目標}

隨著免費遊戲類型的遊戲客戶獲取成本不斷提高,留住玩家成為重要議題。另外,根據網站 Swrve 於 2014 年提供的報告指出,數十款的遊戲中,有 19.3\% 的新玩家只玩一次特定遊戲,新玩家的次日留存率為 33.9\%,而第 30 天的留存率只剩下 5.5\%~\cite{SwrveNewPlayerReport},可以看出新進玩家的流失率極高。因此,若能運用巨量資料探勘框架,來協助預測新進玩家是否有機會流失,將可以針對可能流失玩家作為挽留的目標並進行操作。

本論文的研究目標為預測新進玩家是否流失與了解玩家流失的原因。我們將運用一巨量資料探勘框架,包含對於資料集之前處理,透過特徵工程建立大量特徵,並藉由資料分析方法來探索資料之特性,隨後採用機器學習之分類預測,來預測出可能流失的新進玩家,最後依其機器學習預測結果,分析各個特徵中的突出性,來進行新進玩家流失原因的解釋說明。

\section{研究方法概述}

本論文將新進玩家創帳號後的天數切分為三個時期:(1) 觀察期:玩家創帳號後前幾天,會將此時期的玩家遊戲軌跡作為資料特徵來訓練模型; (2) 挽留期:於觀察期之後,作為給運營操作的時間; (3) 表現期:於挽留期之後,主要決定玩家是否流失,如果玩家在此時期未登入則視為流失玩家,反之視為非流失玩家。

此外,本文還運用一巨量資料探勘框架:此框架將由四大階段組成, (1) 資料前處理階段:首先從資料庫群中整合所有需要的資料,並過濾掉無價值玩家,將觀察期視為資料特徵探勘期,並對其進行特徵工程,將原始特徵透過加總、平均等統計手法來建立新特徵,也額外以天為單位計算來獲得更多特徵,最後著手準備目標值,以利後續分析及機器學習使用; (2) 資料分析階段:使用前階段產出之資料,透過統計圖表來觀察資料特性,進行探索性資料分析 ( Exploratory Data Analysis ),藉由流失玩家與非流失玩家的資料分佈來檢查是否有不合適之資料特徵,並觀察資料特徵是否可以提供給學習模型較多之資訊; (3) 機器學習階段:首先將處理後的資料集分割為訓練集及測試集,隨後針對訓練集進行少數群樣本權重值放大以處理不平衡資料,並透過交叉驗證 ( Cross Validation ) 找出機器學習模型的最佳超參數以獲得最佳模型,其中學習模型選用決策樹 ( Decision Tree )、隨機森林 ( Random Forest ) 及極限梯度提升 ( Extreme Gradient Boosting ),最後藉由測試集來驗證評估最佳模型,產出預測結果; (4) 預測結果分析階段:使用前階段產出之預測結果進行資料特徵重要性分析,透過計算各資料特徵於各學習模型中之 Gini Importance,並搭配決策樹作為代理人模型,以利更加了解及解釋資料特徵與遊戲所提供之體驗綜合評估。

在方法驗證上,本論文將藉由混淆矩陣 ( Confusion Matrix ) 所延伸之接收者操作特征曲線 ( Receiver Operating Characteristic Curve )~\cite{fawcett2006introduction}與精確召回曲線 ( Precision-Recall Curve )~\cite{article}來協助驗證學習模型之優劣,並同時利用 Weighted F$_{\beta}$ - Score~\cite{Goutte2005API}來選出最佳模型與最佳參數解,隨後計算特徵重要性 ( Feature Importance ) 於各資料特徵中,以了解到何者於學習模型中貢獻了最多的資訊量,以利學習模型進行訓練與分類。
\newpage

\section{研究貢獻}

本論文之研究貢獻為:

\begin{enumerate}
    \item 提出一新進玩家觀察期,排除舊有玩家,以利學習模型著重於新進玩家的資訊上。
    \item 提出一新進玩家挽留期,預留時間以方便運營做挽留操作。
    \item 提出一新進玩家表現期,配合新進玩家觀察期,以利學習模型預測玩家是否流失。
    \item 透過資料特徵工程,以統計手法建立更多資料特徵。
    \item 於資料集中進行資料特徵之探勘,藉由不同種類與面向之方式,挑選出適合用來呈現流失玩家的資料。
    \item 整理出適合於不平衡資料集中的評估值方式,將對於學習模型之預測結果提供合理的評估,進而進行比較。
    \item 整理出資料特徵重要性之計算,以利分析資料特徵的突出性與其貢獻的資訊量。
\end{enumerate}

\section{本論文之章節結構}

%本論文第~\ref{cha:RelatedWork}~章為文獻探討,探討關於資料前處理、學習模型選擇、資料不平衡處理以及其評估方式的相關文獻。第~\ref{cha:Method}~章為研究方法,詳述講述各階段之研究流程,並分為四節來介紹資料前處理、資料分析、機器學習及預測結果分析。第~\ref{cha:Evaluation}~章為實驗結果與分析,分為五節:第~\ref{sec:SystemStructure}~節 實驗系統架構;第~\ref{sec:DataPreprocessEvaluation}~節 資料前處理評估;第~\ref{sec:DataAnalysisEvaluation}~節 資料分析評估;第~\ref{sec:MachineLearningEvaluation}~節 機器學習評估;第~\ref{sec:PredictionResultAnalysisEvaluation}~節 預測結果分析評估。第~\ref{cha:Conclusions}~章為結論與未來研究,總結本論文提出的方法與實驗結果,並討論未來的研究方向。
\newpage