\chapter{緒論}

\section{研究背景與動機}

對於許多遊戲商而言,準確預測玩家流失對於長期成功至關重要。近年來,手機遊戲商大多以免費遊戲為主,不再是以往的買斷制或月費制,在免費遊戲的模式之下,遊戲商之營收有非常顯著的成長,例如:「絕地要塞 2」 ( Team Fortress 2 ) 原本為買斷制型式遊戲,新收入的來源僅限於沒購買過遊戲的人,這樣的商業模式與他們的發展策略並不完全吻合。於是,在 2011 年改為免費遊戲,除了擴大玩家受眾,還透過遊戲不斷地更新保持玩家的興趣,最終,遊戲營收提高達 12 倍之多~\cite{miller2012gdc}。但是,免費遊戲在定義是否為流失玩家上極為困難,一旦玩家想離開了隨時都能停止遊玩,沒有必要告訴遊戲商其決定,遊戲商也容易因此失去挽回玩家的機會。

根據哈佛商業評論~\cite{HarvardBusiness}中指出,企業爭取一個新顧客的成本是保留老顧客成本的 5 倍,一個公司如果能將其顧客流失率降低 5\%,其利潤就能增加 25\% - 85\%。也就是說,獲得一個新玩家的成本比留住一個玩家要昂貴許多。因此,留住玩家成為重要議題,精準地預測玩家流失,即使是微小地提升,也可能導致營收有顯著的提升。

另外,根據網站 Swrve~\cite{SwrveNewPlayerReport} 於 2014 年提供的報告指出,數十款的遊戲中,有 19.3\% 的新玩家只玩一次特定遊戲,新玩家的次日留存率為 33.9\%,而第 30 天的留存率只剩下 5.5\%。而 Mustač 等人~\cite{SupervisedMachineLearning}則是針對歐洲一款休閒遊戲進行研究,研究中可以看到,此遊戲有大約 60\% 的玩家在玩了一天後就離開了,3 天後,玩家們只剩下大約 20\% 。上述的 2 份報告都清楚表明了新進玩家的流失率是很嚴重的問題,因此,本文將針對新進玩家做流失預測分析,希望能因此提高新進玩家的留存率,進而有效地增加營收。

\section{研究目標}

本論文的研究目標是預測免費遊戲的新進玩家是否會流失。透過新進玩家在遊戲初期的遊玩歷程和儲值紀錄等特徵訓練出準確率最高的模型,並藉由模型的結果分析個個特徵的突出性,來進行玩家流失原因的解釋說明,甚至能從中了解玩家們的喜好、趨勢,也可以讓市場操作人員有挽留玩家的操作依據,有了明確的挽留策略就能夠對症下藥,藉此來強化留存,並進一步增加營收。

\section{研究方法概述}

本論文將新進玩家創帳號後的天數切分為三個時期:(1) 觀察期:玩家創帳號後前幾天,會將此時期的玩家行為軌跡作為資料特徵來訓練模型; (2) 挽留期:於觀察期之後,作為給市場操作人員實施挽留策略的時間; (3) 表現期:於挽留期之後,決定玩家是否流失,如果玩家在此時期有任一登入紀錄則視為非流失玩家,反之將視為流失玩家。

此外,本文還運用一巨量資料探勘框架:此框架將由四大階段組成, (1) 資料前處理階段:首先從資料庫群中整合所有需要的資料,並過濾掉無價值玩家,再著手目標值準備、資料特徵探勘與特徵工程,以利後續分析及機器學習使用; (2) 資料分析階段:使用前階段產出之資料,透過統計圖表來觀察資料特性,進行探索性資料分析 ( Exploratory Data Analysis )~\cite{tukey1977exploratory},藉由流失玩家與非流失玩家的資料分佈來檢查是否有不合適之資料特徵,並觀察資料特徵是否可以提供給學習模型較多之資訊; (3) 機器學習階段:首先將處理後的資料集分割為訓練集及測試集,隨後針對訓練集進行少數群樣本權重值放大以處理不平衡資料,並透過交叉驗證 ( Cross Validation ) 找出機器學習模型的最佳超參數以獲得最佳模型,其中學習模型選用決策樹 ( Decision Tree )、隨機森林 ( Random Forest ) 及極限梯度提升 ( Extreme Gradient Boosting ),最後藉由測試集來驗證評估最佳模型,產出預測結果; (4) 預測結果分析階段:使用前階段產出之預測結果進行資料特徵重要性分析,透過計算各資料特徵於各學習模型中之基尼重要性 ( Gini Importance ),以利更加了解及解釋資料特徵與遊戲所提供之體驗綜合評估。

在方法驗證上,本論文將藉由混淆矩陣 ( Confusion Matrix ) 所延伸之接收者操作特征曲線 ( Receiver Operating Characteristic Curve )~\cite{fawcett2006introduction}與精確召回曲線 ( Precision-Recall Curve )~\cite{article}來協助驗證學習模型之優劣,並同時利用 Weighted F$_{\beta}$ - Score~\cite{Goutte2005API}來選出最佳模型與最佳參數解,隨後計算特徵重要性 ( Feature Importance ) 於各資料特徵中,以了解到何者於學習模型中貢獻了最多的資訊量,以利學習模型進行訓練與分類。

\section{研究貢獻}

本論文之研究貢獻為:

\begin{enumerate}
    \item 訓練一新進玩家流失預測模型並運用於遊戲領域,利用其預測結果從中了解玩家流失的原因與動機,並作為市場操作人員後續挽留玩家的操作依據,以強化留存並進一步增加營收。
    \item 提出一資料特徵工程的方法:對資料集以統計手法建立資料特徵,並用多個時間框架做拆分,以獲得第一層特徵變數,再對第一層特徵變數做計算,進一步來得到第二層特徵變數,如變化量特徵等。
    \item 於資料集中進行資料特徵之探勘,藉由不同種類與面向之方式,挑選出適合用來呈現流失玩家的資料。
    \item 整理出適合於不平衡資料集中的評估值方式,將對於學習模型之預測結果提供合理的評估,進而進行比較。
    \item 整理出資料特徵重要性之計算,以利分析資料特徵的突出性與其貢獻的資訊量。
\end{enumerate}

\section{本論文之章節結構}

%本論文第~\ref{cha:RelatedWork}~章為文獻探討,探討關於資料前處理、學習模型選擇、資料不平衡處理以及其評估方式的相關文獻。第~\ref{cha:Method}~章為研究方法,詳述講述各階段之研究流程,並分為四節來介紹資料前處理、資料分析、機器學習及預測結果分析。第~\ref{cha:Evaluation}~章為實驗結果與分析,分為五節:第~\ref{sec:SystemStructure}~節 實驗系統架構;第~\ref{sec:DataPreprocessEvaluation}~節 資料前處理評估;第~\ref{sec:DataAnalysisEvaluation}~節 資料分析評估;第~\ref{sec:MachineLearningEvaluation}~節 機器學習評估;第~\ref{sec:PredictionResultAnalysisEvaluation}~節 預測結果分析評估。第~\ref{cha:Conclusions}~章為結論與未來研究,總結本論文提出的方法與實驗結果,並討論未來的研究方向。
\newpage