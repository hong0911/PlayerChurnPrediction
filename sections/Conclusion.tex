\chapter{結論與未來研究}
\label{cha:Conclusions}

\section{結論}

本文提出一巨量資料探勘框架,此框架由五大階段組成:資料前處理、資料分析、機器學習、預測結果分析及產業應用分析。於資料前處理階段進行資料的整合與過濾,以篩選出有價值玩家資料,來提高整體分析成效,接著進行目標值準備、資料特徵探勘與特徵,透過一時間框架,定義流失及非流失玩家,並建立多個特徵變數,使得後續機器學習更加順利;於探索性資料分析階段分析資料特徵,提早推測資訊量較高的特徵;於機器學習階段預測結果,在訓練機器之前會調整權重值來解決資料不平衡問題;於預測結果分析階段計算特徵重要性,可整理出資料特徵突出的原因,例如:玩家登入天數等;最後於產業應用分析階段了解流失玩家的行為規則,可以做為市場操作人員實施挽留策略的依據。

此框架可以應用於遊戲領域且著重於新進玩家。透過資料特徵重要性分析,可以發現本文的第二層特徵變數\ (\ 即變化量特徵\ )\ 確實對模型的預測上有很大的幫助,讓實驗上有極好的數值表現;也能透過代理人模型快速地知道玩家流失的原因,藉由準確地採取策略,以強化遊戲玩家的留存,並進一步提高營收。

\section{未來研究}

由於本文探勘的資料特徵種類受限於遊戲平台所提供的原始資料集,如能獲得更進一步的詳細資料,或是盡可能地擴增特徵變數,將可更加準確的預測出流失玩家。此外,本文選擇的學習模型只有三種樹狀結構之模型,未來可進行更多實驗於不同類型的學習模型。於模型驗證上,未來也可以透過 A/B 測試,觀察哪個模型的結果比較可以提高玩家的留存率。雖然關於流失預測已有許多研究,但於產業應用上尚有許多議題可以探討。