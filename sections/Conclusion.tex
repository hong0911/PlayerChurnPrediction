\chapter{結論與未來研究}
\label{cha:Conclusions}

\section{結論}

本論文提出一巨量資料探勘框架,框架拆分為四大階段進行,其中包括資料前處理、資料分析以及機器學習訓練等步驟。透過此框架可了解到資料前處理中的整合資料需求,並在清理資料時,過濾掉無價值的玩家,以提高整體分析成效,並且在目標值準備與資料特徵探勘時,給予一時間門檻,定義出合理的資料集,使得後續機器學習更加順利;藉由探索性資料分析處理資料集中不合適的資料特徵,並提早推測有助於遊戲平台發展的資料特徵;經由加入權重值於少數群,解決遊戲領域遭遇到的資料不平衡問題,而不透過修改原資料集的內容,再利用學習模型所輸出的預測結果,預測出付費玩家;最後以資料特徵重要性分析與藉由前述之推測,整理出資料特徵突出之原因,例如:玩家於遊戲體驗之起伏、玩家獲得獎勵或贏得遊戲所提高其付費意願。我們提出的框架可應用於遊戲領域且著重於新進玩家並經實驗證實,藉由前處理後的資料集,並同時針對資料不平衡進行處理,在預測潛在之新進付費玩家上有不錯的表現!並且我們可以進一步分析各資料特徵的重要性,協助解釋預測結果與遊戲內玩家行為軌跡的連動性,將可在遊戲平台行銷策略上提供意見,以更符合玩家真實環境所需,提高往後玩家之消費意願。

\section{未來研究}

由於本論文的資料特徵探勘種類受限於遊戲平台所提供的原始資料集,如能獲得更進一步的詳細資料,將可更加準確的預測出付費玩家。此外我們使用到的學習模型只有三種樹狀結構之模型,未來將可進行更多實驗於不同類型的學習模型,甚至導入時間序的概念,了解到玩家遊玩遊戲的順序差異是否會影響消費意願。最後除了可預測玩家是否會付費之外,未來還可進行付費時間點、購買商品種類等之預測,尚有許多議題可於遊戲領域巨量資料中研究。